\section{Introduction: Motivation for Text Processing}

Vast amounts of knowledge are trapped in presentation media such as videos, html, pdfs, and paper as opposed to being concept-mapped, interlinked, addressable and reusable at fine grained levels. This defeats knowledge exchanges between humans and between human cognition and AI-based systems.

It is known that concept mapping enhances human cognition. Especially in domain-specific areas of knowledge, better interlinking would be achieved if concepts would be extracted using surrounding context. ``You shall know a word by the company it keeps" (Firth, 1957). However, in order to scale the benefits, we need to extend this activity to a large base of knowledge. To do that effectively, we can use  machine learning and natural language processing to ensure that the meaning of the concepts are:

1. statistically relevant in the body of knowledge

2. provably grounded in the resources and the knowledge of the participants (authors, users)

3. evolving with time (called dynamic word embeddings)

Without these features, we cannot expect participants to adopt the terms and concepts.

